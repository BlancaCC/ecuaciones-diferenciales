\documentclass[a4paper , 14pt, spanish ]{article}

% Codificación e idioma, para las tildes crucial
\usepackage[utf8]{inputenc}
\usepackage[spanish]{babel}
\usepackage[T1]{fontenc}
\usepackage{textcomp}

%paquetes para matemáticas
\usepackage{fullpage,mathpazo,amsfonts,nicefrac, amssymb , amsmath, amsbsy}
\usepackage{amsmath}
%paquetes para insertar gráficas (imágenes)
\usepackage{graphicx}

%paquete de enumerar, para realizar listas 
\usepackage{enumerate}

\usepackage{import}

%%%%%%%%%%%%%%%%%%%%%%%%%%%%%%%%%% formato chulo %%%%%%%%%%%%%%%%%%%%%%%%%%%%%



%%%%%%%%%%%%%%%%%%%%%%%%%%%%%%%%%%%%%%55 cuadros bonitos %%%%%%%%%%%%%%%%%%%%%%%%%%%%%%%%%%%
\usepackage{cleveref}
\usepackage[most]{tcolorbox}
\newtcbtheorem{Theorem}{Theorem}{
  enhanced,
  sharp corners,
  attach boxed title to top left={
    yshifttext=-1mm
  },
  colback=white,
  colframe=blue!75!black,
  fonttitle=\bfseries,
  boxed title style={
    sharp corners,
    size=small,
    colback=blue!75!black,
    colframe=blue!75!black,
  } 
}{thm}
\newtcbtheorem[no counter]{Proof}{Ejercicio}{
  enhanced,
  sharp corners,
  attach boxed title to top left={
    yshifttext=-1mm
  },
  colback=white,
  colframe=blue!25,
  fonttitle=\bfseries,
  coltitle=black,
  boxed title style={
    sharp corners,
    size=small,
    colback=blue!25,
    colframe=blue!25,
  } 
}{prf}




%\documentclass[a4paper]{article} 
%%

\begin{document}

%-------------------------------
%	TITLE SECTION
%-------------------------------

%\fancyhead{}
%%\hrule\medskip % Upper rule
%%\begin{minipage}{0.295\textwidth} 

%%\end{minipage}
%%\begin{minipage}{0.4\textwidth} 
%\centering 
%\large 
%Tema 1%\\ 
%\normalsize

%\flushright
%%\footnotesize
%Blanca Cano Camarero %%\hfill %\\   


%\end{minipage}
%\begin{minipage}{0.295\textwidth} 
%\flushleft
%\today%%\hfill%\\
%\end{minipage}
%\medskip\hrule 
%\bigskip


\raggedright
%%%%%%%%%%%%%%%%%%%%%%%%%%%%%%%%%%%%%%%%%%%%%%%%%%%%%%%%%%%%%%%%%%%%%%%%%%%%%%%%%%%%%%%%%%%%%%%%%%%%%%%
\title{Ejercicio extra teórico tema 1}
\author{Blanca Cano Camarero}

%\begin{document}
%\maketitle

\begin{table}[ht]
  \resizebox{\textwidth}{!}{
    \small
\begin{tabular}[t]{ccc}
  Blanca Cano Camarero & Ecuaciones diferenciales II & \today \\
  \hline
 
\end{tabular}}

\end{table}%
%%%%%%%%%%%%%%%%%%%%%%%%%%%%%%%%%

\setlength{\parindent}{10ex} %% tabulador al principio

\begin{Proof}{Extra teórico tema 1}{}

  Sea $A: \mathbb R \longrightarrow \mathcal {M}_d (\mathbb R )$ continua y $T$-periódica dondde $T>0$
  y sea $\phi: \mathbb R \longrightarrow  \mathcal {M}_d (\mathbb R )$ la matriz fundamental principal en $t=0$
  de la EDO lineal homogénea
  $$(*) x' = A(t) x. $$


  Prueba:

  \begin{enumerate}
  \item Prueba que existe una matriz $M$, llamada matriz de monondromía, tal que $\phi(t+T) = \phi(t) \dot M$ para todo $t \in \mathbb R.$

  \item Prueba que (*) es asintóticamente estable sii $ \lim_{ n \longrightarrow \infty} M^n = 0.$

  \item Prueba que (*) es estable sii $\{ M^n : n \in \mathbb N\}$ es acotada.

  \item Los valores propios de $M$ reciben el nombre de multiplicadores de Floquet ¿Puedes caracterizar la estabilidad, inestabilidad y estabilidad asintótica
    de (*) en función de los multiplicadores de Floque?
    

    
    \end{enumerate}
 
\end{Proof}


\subsection{Existencia matriz de monodromía }


Por ser matriz fundamental sabemos que es invertible y además  cumple que $\phi'(t) = A(t) \phi(t)$
despejamos  $A(t) = \phi'(t)  \phi^{-1}(t)  \; (1).$\\

Utilizando ahora la $T-$periodicidad   tenemos que $\phi'(t+T) = A(t+T) \phi(t+T) =  A(t) \phi(t+T) \; (2).$ \\

Sustituyendo el valor de $A(t)$ de (1) en (2) llegamos a que :

$$\phi'(t+T) =  \phi'(t)  \phi^{-1}(t)  \phi(t+T)$$

Multiplicando a la izquierda por la inversa de $\phi'(t)  \phi^{-1}(t)$

$$\phi(t) {\phi'(t)}^{-1}    \phi'(t+T) =   \phi(t+T)$$

Si llamamos $M =  {\phi'(t)}^{-1}    \phi'(t+T)$ acabamos de encontrar la matriz de monodromía buscada. 





\subsection{ Prueba que (*) es asintóticamente estable sii $\lim_{ n \longrightarrow \infty} M^n = 0.$}


Veamos primero la siguiente observación: \\
Cuando $t$ es lo suficientemente grande, podemos escribir $t = t_0+ nT$ con $t_0 \in [0, T)$ y $n \in \mathbb N.$

De donde podemos ver que:

$$\phi(t) =  \phi(t_0 + nT) = \phi(t_0 + (n-1)T) M = \phi(t_0) M^n$$

Recordemos ahora caracterización equivalente de asintóticamente estable:

(*) será asintóticamente estable (un atractor) si la matriz fundamente de (**) converge hacia 0 cuantdo $t$ tiende a infinito.

Como no tiene porqué darse el caso trivial de que  $\phi(t_0) = 0$ para todo $t_0 \in [0, T)$,  entonces necesariamente (y suficientemente)   $\lim_{ n \longrightarrow \infty} M^n = 0$ sii $\lim_{t \longrightarrow}\phi(t) = 0$

 

\subsection { Prueba que (*) es estable sii $\{ M^n : n \in N \}$ es acotada.}

Recordemos que por las equivalencias de estabilidad,  (*) será estable sii la matriz fundamental está acotada en $[0, \infty)$
  (La estabilidad  es invariante del punto seleccionado como extremo inferior, por ello he seleccionado el 0).   \\


Sabemos que  $\phi(t_0)$ está acotada en $t \in [0, T]$ por ser compacto. \\

Teniendo presente la observación anterior: \\

Para $t> 0$  podemos escribir  $t = t_0+ nT$ con $t_0 \in [0, T)$ y $n \in \mathbb N.$

$$\phi(t) = \phi(t_0) M^n$$

Ya hemos visto que $\phi(t_0)$ está acotada, luego  $\phi(t)$ estará acotada si y solo si $\{ M^n : n \in \mathbb N\}$ lo está.


\subsection{ Caracterización de  la estabilidad, inestabilidad y estabilidad asintótica  de (*). }

En los apartado 2 y 3 de la demostración ya hemos caracterizado la estabilidad y la estabilidad asintótica en función de $M.$

A partir de sus valores propios podemos ver $M = P J P^{-1}$ donde $J$  es una matriz de Jordan y $P$ su respectiva matriz de paso.
Además $M^n = P J^n P^{-1}.$

Podemos pues ligar el compartamiento asintótico de $M^n$ al de $J^n$ que depende exclusivamente de los valores propios de $M.$

Recordemao que si $J_m$ una caja de jordan de la forma:  \\
\[
J_m =
\begin{bmatrix}
  a & 1 & ...& .. &  0\\
  0 & a & 1 & ... &  0\\
  &  & ... & & \\
  0 & ... &..& 0 & a
\end{bmatrix}_{m \times m}	
\]

Entonces con $\mathcal{X}_{n-(m-fila)>0}$  la función característica. 

\[
J_m^n =
\begin{bmatrix}
  a^n &  a^{n-1} n && .. & \frac{a^{n-(m-1)} (n+m-fila)!}{(m-1)!} \mathcal{X}_{n-(m-fila)>0} \\
  0 & a^n & a^{n-1} n  & ... &  \frac{a^{n-(m-fila)} (n+m-fila)! }{(m-fila)!} \mathcal{X}_{n-(m-fila)>0} \\
  &  & ... & & \\
  0 & ... &..& 0 & a^n
\end{bmatrix}_{m \times m}	
\]

Asintóticamente los componentes de la matriz triangular superior dependen esclusivamente de $a$.


Finalemente y para estudiar el caso complejo podemos pensar los valores en su forma polar $\lambda = r e^{i \theta}$ y está claro que
$\lambda ^n  = r^n e^{i n \theta}.$ Donde $r = |\lambda|$ es el módulo,  $\theta$ el ángulo que forman y $n$ un número natural. Como la exponencial compleja nunca se anula, necesariamente el valor asíntótico también dependerá del módulo. 





\begin{enumerate}

\item \textbf{Estabilidad asintótica.} Si $M^n$ tiende a la matriz $0$ cuando $n$ tiende a infinito o equivalentemente si  $J^n$ tiende
  a la matriz  $0$ cuando $n$ tiende a infinito. \\
  Esto será si $| \lambda | < 1$  para todo $\lambda \in \sigma (M).$

  \item \textbf{Estabilidad} Si $\{ M^n : n \in N \}$ está acotada o quivalentemente si  $\{ J^n : n \in N \}$ está acotada.  \\
    Esto será si $ |\lambda| \leq 1$  para cualquier $\lambda \in \sigma (M).$

  \item \textbf{ Inestabilidad} Teniendo en cuenta los casos anteriores esto será si existe un valor propio de $M$ con módulo estrictamente mayor que uno. 

  
\end{enumerate}




\end{document}
