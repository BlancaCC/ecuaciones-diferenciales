\documentclass[a4paper , 14pt, spanish ]{article}

% Codificación e idioma, para las tildes crucial
\usepackage[utf8]{inputenc}
\usepackage[spanish]{babel}
\usepackage[T1]{fontenc}
\usepackage{textcomp}

%paquetes para matemáticas
\usepackage{fullpage,mathpazo,amsfonts,nicefrac, amssymb , amsmath, amsbsy}
\usepackage{amsmath}
%paquetes para insertar gráficas (imágenes)
\usepackage{graphicx}

%paquete de enumerar, para realizar listas 
\usepackage{enumerate}

\usepackage{import}

%%%%%%%%%%%%%%%%%%%%%%%%%%%%%%%%%% formato chulo %%%%%%%%%%%%%%%%%%%%%%%%%%%%%



%%%%%%%%%%%%%%%%%%%%%%%%%%%%%%%%%%%%%%55 cuadros bonitos %%%%%%%%%%%%%%%%%%%%%%%%%%%%%%%%%%%
\usepackage{cleveref}
\usepackage[most]{tcolorbox}
\newtcbtheorem{Theorem}{Theorem}{
  enhanced,
  sharp corners,
  attach boxed title to top left={
    yshifttext=-1mm
  },
  colback=white,
  colframe=blue!75!black,
  fonttitle=\bfseries,
  boxed title style={
    sharp corners,
    size=small,
    colback=blue!75!black,
    colframe=blue!75!black,
  } 
}{thm}
\newtcbtheorem[no counter]{Proof}{Ejercicio}{
  enhanced,
  sharp corners,
  attach boxed title to top left={
    yshifttext=-1mm
  },
  colback=white,
  colframe=blue!25,
  fonttitle=\bfseries,
  coltitle=black,
  boxed title style={
    sharp corners,
    size=small,
    colback=blue!25,
    colframe=blue!25,
  } 
}{prf}




%\documentclass[a4paper]{article} 
%%

\begin{document}

%-------------------------------
%	TITLE SECTION
%-------------------------------

%\fancyhead{}
%%\hrule\medskip % Upper rule
%%\begin{minipage}{0.295\textwidth} 

%%\end{minipage}
%%\begin{minipage}{0.4\textwidth} 
%\centering 
%\large 
%Tema 1%\\ 
%\normalsize

%\flushright
%%\footnotesize
%Blanca Cano Camarero %%\hfill %\\   


%\end{minipage}
%\begin{minipage}{0.295\textwidth} 
%\flushleft
%\today%%\hfill%\\
%\end{minipage}
%\medskip\hrule 
%\bigskip


\raggedright
%%%%%%%%%%%%%%%%%%%%%%%%%%%%%%%%%%%%%%%%%%%%%%%%%%%%%%%%%%%%%%%%%%%%%%%%%%%%%%%%%%%%%%%%%%%%%%%%%%%%%%%
\title{Ejercicio 7 tema 1}
\author{Blanca Cano Camarero}

%\begin{document}
%\maketitle

\begin{table}[ht]
  \resizebox{\textwidth}{!}{
    \small
\begin{tabular}[t]{ccc}
  Blanca Cano Camarero & Ecuaciones diferenciales II & \today \\
  \hline
 
\end{tabular}}

\end{table}%
%%%%%%%%%%%%%%%%%%%%%%%%%%%%%%%%%

\setlength{\parindent}{10ex} %% tabulador al principio

\begin{Proof}{7 tema 1}{}
Demuestra que la propiedad de estabilidad de una ecuación lineal es independiente del instante fijado $t_o$.
\end{Proof}


Sea $\varphi: (\alpha , + \infty) \rightarrow \mathbb R^d$ una  solución  estable en  $t_0 \in (\alpha, +\infty)$ de la ecuación lineal
$x' = A(t) x + b(t)$ (*).  \\ 

 La definición dada en teoría de estabilidad es la siguiente:  \\

Para todo $\epsilon > 0$ existirá  un $\delta > 0$ tal que para cualquier
otra solución 
$\mathcal X  : (\alpha , + \infty) \rightarrow \mathbb R^d$ de la ecuación (*) que cumpla que   $\| \varphi(t_o) - \mathcal X(t_0) \| < \delta$,
se tendrá entonces que   $\| \varphi(t) - \mathcal X(t) \| < \epsilon$ para  $t>t_0$ . \\

Nótese que con dicha definición, a priori la estabilidad depende de $t_o$, vemos ahora que es independiente. \\



En virtud de una de las equivalencias de caracterización de estabilidad, sabemos que  si
$\varphi : (\alpha , + \infty) \rightarrow \mathbb R^d$ es solución estable para $t_0 \in (\alpha, +\infty)$  entonces
para cualquier solución $y : (\alpha , + \infty) \rightarrow \mathbb R^d$ de la ecuación homogénea $y' = A(t) y$ se cumple que está acotada en
$[t_o, \infty)$.

Veamos ahora que para un instante  $t_1 \in (\alpha, +\infty)$ arbitario  existirá $M \in \mathbb R$ cumpliendo  que 
$\|y(s)\|< M$ para cualquier $s \in [t_1, +\infty)$, es decir que está acotada y por ende es estable para $t_1$.

Distingamos los siguientes casos:

\begin{enumerate}

\item Si $t_o \leq t_1$ se tiene que $[t_1, +\infty) \subseteq [t_o, +\infty)$ para el cual está acotada por hipótesis, luego no habría nada que probar.

\item Si $t_1 < t_0$ entonces por el razonamiento anterior sabríamos que está acotado en $[t_o, +\infty)$, faltando por comprabar que también lo está en el
  compacto $[t_1, t_0]$. \\

  Conocemos además que $y$ es continua,  la norma también y la composición de continuas es continua. \\

  Toda función continua tiene  máximo dentro de un compacto, es decir existe $M \in \mathbb R$ para el cual $\|y(s)\|< M$ sea cual sea $s \in [t_1, t_0]$. \\

  Concluímos por tanto que estaría acotada en $[t_1, t_0]$  por compacidad y en $[t_o, +\infty)$ por la hipótesis de estabilidad en $t_0$, probando con esto la estabilidad en $t_1.$ \\

    Como $t_1$ era arbitrario queda demostrado que la estabilida no depende del instante inicial de la demostración. 

    

  
  
\end{enumerate}






\end{document}
